% ---------
%  Compile with "pdflatex hw0".
% --------
%!TEX TS-program = pdflatex
%!TEX encoding = UTF-8 Unicode

\documentclass[11pt]{article}
\usepackage{jeffe,handout,graphicx}
\usepackage[utf8]{inputenc}		% Allow some non-ASCII Unicode in source

%  Redefine suits
\usepackage{pifont}
\usepackage{algorithm}
\usepackage{algorithmic}
\usepackage{amsmath}

\def\Spade{\text{\ding{171}}}
\def\Heart{\text{\textcolor{Red}{\ding{170}}}}
\def\Diamond{\text{\textcolor{Red}{\ding{169}}}}
\def\Club{\text{\ding{168}}}

\def\Cdot{\mathbin{\text{\normalfont \textbullet}}}
\def\Sym#1{\textbf{\texttt{\color{BrickRed}#1}}}


% =====================================================
%   Define common stuff for solution headers
% =====================================================
\Class{CS $450$}
\Semester{Spring $2019$}
\Authors{1}
%\Section{}

% =====================================================
\begin{document}

% ---------------------------------------------------------


% ---------------------------------------------------------
% Change authors again
\AuthorOne{Ray Ying}{xinruiy2@illinois.edu}


\HomeworkHeader{$4$}{$1$}

\begin{quote}

\end{quote}
\hrule


\begin{solution}
\item
\begin{enumerate}
    \item   Since we know $A$ is non-singular, we can show that $\norm{Ax}_2 \cdot \norm{A^{-1}}_2 = \norm{A^{-1}}_2 \cdot \norm{Ax}_2 \geq \norm{A^{-1} \cdot Ax}_2 = \norm{x}_2$. \\
    Hence, $\norm{Ax}_2 \cdot \norm{A^{-1}}_2 \geq \norm{x}_2 \implies \frac{1}{\norm{A^{-1}}_2} \cdot \norm{x}_2 \leq \norm{Ax}_2$. 
    
    \item   Take the right side of the inequality, we can first show that $\norm{A-B}_2 \cdot \norm{x}_2 + \norm{Bx}_2 \geq \norm{(A-B)x}_2 + \norm{Bx}_2 = \norm{Ax - Bx}_2 + \norm{Bx}_2$. \\
    From triangular inequality, $\norm{Ax - Bx}_2 \geq \norm{Ax}_2 - \norm{Bx}_2 \implies \norm{Ax - Bx}_2 + \norm{Bx}_2 \geq \norm{Ax}_2$.\\
    Hence, we can show that $\norm{A-B}_2 \cdot \norm{x}_2 + \norm{Bx}_2 \geq \norm{Ax - Bx}_2 + \norm{Bx}_2 \geq \norm{Ax}_2 \implies \norm{A-B}_2 \cdot \norm{x}_2 + \norm{Bx}_2 \geq \norm{Ax}_2$.
    
    \item   $k_2(A) = \norm{A^{-1}}_2 \cdot \norm{A}_2$, then we can have the following:
    $$ \frac{\norm{A-B}_2}{\norm{A}_2} < \frac{1}{\norm{A^{-1}}_2 \cdot \norm{A}_2} $$ 
    Since $\norm{x}_2 \geq 0$, we can multiply $\norm{x}_2$ on both side of the inequality and divide $\norm{A}_2$. We will have $$ \norm{A-B}_2 \cdot \norm{x}_2 < \frac{\norm{x}_2}{\norm{A^{-1}}_2}$$
    Based on what we prove in part $a$, $\frac{\norm{x}_2}{\norm{A^{-1}}_2} \leq \norm{Ax}_2$. Similar, based on part $b$, we have $\norm{A-B}_2 \cdot \norm{x}_2 \geq \norm{Ax}_2 - \norm{Bx}_2$. Therefore, we can rewrite the inequality to:
    $$ \norm{Ax}_2 - \norm{Bx}_2 < \norm{Ax}_2 $$
    In order this to be true, we want to have $Bx = 0$ only have a trivial solution, when $x = 0$. \\
    Hence, we can prove that if we have $A$ non-singular and $\frac{\norm{A-B}_2}{\norm{A}_2} < \frac{1}{K_2(A)}$, then $B$ is non-singular.
        
    \end{enumerate}
\end{solution}
\end{document}
