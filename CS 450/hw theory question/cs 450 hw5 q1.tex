% ---------
%  Compile with "pdflatex hw0".
% --------
%!TEX TS-program = pdflatex
%!TEX encoding = UTF-8 Unicode

\documentclass[11pt]{article}
\usepackage{jeffe,handout,graphicx}
\usepackage[utf8]{inputenc}		% Allow some non-ASCII Unicode in source

%  Redefine suits
\usepackage{pifont}
\usepackage{algorithm}
\usepackage{algorithmic}
\usepackage{amsmath}
\documentclass{article}
\usepackage{amsmath}

\def\Spade{\text{\ding{171}}}
\def\Heart{\text{\textcolor{Red}{\ding{170}}}}
\def\Diamond{\text{\textcolor{Red}{\ding{169}}}}
\def\Club{\text{\ding{168}}}

\def\Cdot{\mathbin{\text{\normalfont \textbullet}}}
\def\Sym#1{\textbf{\texttt{\color{BrickRed}#1}}}


% =====================================================
%   Define common stuff for solution headers
% =====================================================
\Class{CS $450$}
\Semester{Spring $2019$}
\Authors{1}
%\Section{}

% =====================================================
\begin{document}

% ---------------------------------------------------------


% ---------------------------------------------------------
% Change authors again
\AuthorOne{Ray Ying}{xinruiy2@illinois.edu}


\HomeworkHeader{$5$}{$1$}

\begin{quote}

\end{quote}
\hrule


\begin{solution}
\item
    \begin{enumerate}
    \item  $A + uv^T = Q(R + wv^T) $, since $A = QR$\\
    $QR + uv^T = QR + Qwv^T \implies uv^T = Qwv^T \implies Q^Tuv^T = Q^TQwv^T$\\
    $Q$ is a orthogonal matrix, $Q^TQ = I$\\
    $Q^Tuv^T = wv^T \implies Q^Tu = w$
    \item $R + wv^T = Q_1(Q_1^TR + ce_1v^T)$, $Q_1$ is orthogonal because it is the product of orthogonal matrices(a series of Givens rotations).\\
    $R + wv^T = R + Q_1ce_1v^T \implies wv^T = Q_1ce_1v^T$\\
    Therefore, we want to show that $Q_1^Tw = ce_1$. This requires $Q_1^T$ to zeroing out every entry in $w$ except the first entry. Therefore, we will apply a series of rotations in this order $G_{0,1}G_{1,2}...G_{n-1,n}$ to achieve it. $Q_1$ will be $G_{n-1,n}^TG_{n-2,n-1}^T...G_{0,1}^T$. \\
    It's easy to show $wv^T$ is upper triangular since $ce_1v^T$ is upper triangular.(Only the first row has value that's not $0$), therefore, $wv^T = Q_1ce_1v^T$ is also upper triangular. 
    \item We want to show that $Q_1(Q_1^TR + ce_1v^T) = Q_1(H)$, then $Q_1^TR + ce_1v^T = H$.\\
    We already show that $ce_1v^T$ is upper triangular since its value is only in the first row, we want to show that $Q_1^TR$ is upper Hessenberg matrices.\\
    From wikipedia, an upper Hessenberg matrix multiply an upper triangular matrix is still upper Hessenberg matrix. $R$ is a upper triangular matrix, so we only need to prove that $Q_1^T$ is a upper Hessenberg matrix.\\
    $Q_1^T = G_{0,1}G_{1,2}...G_{n-1,n}$, since each matrix $G_{k-1,k}$ only has one nonzero entry in the $[k,k-1]$ location below the diagonal in the matrix. Therefore, for different $G_{k-1,k}$ matrix, the nonzero entries below the diagonal are in different location and they are all in the first off-diagonal below the diagonal. Therefore, the product of $ G_{0,1}G_{1,2}...G_{n-1,n}$ will still only have nonzero value on and above the diagonal, or on the first off-diagonal below the diagonal. \\
    Hence, $Q_1^T$ is Upper Hessenberg matrix, implies $Q_1^TR + ce_1v^T$ is Upper Hessenberg matrix. Therefore, $Q_1(Q_1^TR + ce_1v^T) = Q_1(H)$. 
    \item $Q_1 = G_{n-1,n}^TG_{n-2,n-1}^T...G_{0,1}^T$, $H$ is an Upper Hessenberg matrix, we want $Q_1(H)$ to be an upper triangular. The transpose of a rotation will still be a rotation but in the opposite direction. Since the rotation matrix in the class is \\
    \[
      \begin{bmatrix}
        c & s\\
        -s & c
      \end{bmatrix}
    \]
    The transpose will be 
    \[
      \begin{bmatrix}
        c & -s\\
        s & c
      \end{bmatrix}
    \]
    Indeed, it is doing the zeroing out on the transpose of $H$, which will zeroing out all just the values on the first off-diagonal below the diagonal of the Upper Hessenberg matrix. Therefore, $Q_1(H)$ will be an upper triangular matrix.
    \item The total time for the above process is computing the new $R$, since the rotation matrix $Q_1$ multiply a vector takes $O(n)$ and  the rotation matrix $Q_1$ multiply a matrix takes $O(n^2)$. Therefore, the total time will be the leading term $O(n^2)$.
    \end{enumerate}
\end{solution}
\end{document}
