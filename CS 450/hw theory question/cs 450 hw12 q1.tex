% ---------
%  Compile with "pdflatex hw0".
% --------
%!TEX TS-program = pdflatex
%!TEX encoding = UTF-8 Unicode

\documentclass[11pt]{article}
\usepackage{jeffe,handout,graphicx}
\usepackage[utf8]{inputenc}		% Allow some non-ASCII Unicode in source

%  Redefine suits
\usepackage{pifont}
\usepackage{algorithm}
\usepackage{algorithmic}
\usepackage{amsmath}
\documentclass{article}
\usepackage{amsmath}

\def\Spade{\text{\ding{171}}}
\def\Heart{\text{\textcolor{Red}{\ding{170}}}}
\def\Diamond{\text{\textcolor{Red}{\ding{169}}}}
\def\Club{\text{\ding{168}}}

\def\Cdot{\mathbin{\text{\normalfont \textbullet}}}
\def\Sym#1{\textbf{\texttt{\color{BrickRed}#1}}}


% =====================================================
%   Define common stuff for solution headers
% =====================================================
\Class{CS $450$}
\Semester{Spring $2019$}
\Authors{1}
%\Section{}

% =====================================================
\begin{document}

% ---------------------------------------------------------


% ---------------------------------------------------------
% Change authors again
\AuthorOne{Ray Ying}{xinruiy2@illinois.edu}


\HomeworkHeader{$12$}{$1$}

\begin{quote}

\end{quote}

\hrule

\begin{solution}
\item 

\begin{enumerate}
    \item 
        \begin{enumerate}
            \item Because $f(x) < 0$ for all $x \in [a,b]$ ($f$ is continues), let $x^*$ be the point where $f(x^*) \geq f(x)$ for all $x \in [a,b]$ where $x^* \in [a,b]$. Therefore, $\int_{a}^{b} f(x)dx \leq f(x^*)\cdot|b-a|$. $f(x^*) < 0$, thus $\int_{a}^{b} f(x)dx < 0$\\
            
            \item $\int_{a}^{b} p(x)x^kdx = 0$, we know that $k = 1, ... ,n-1$, the sign of $x^k$ can be negative or positive. Therefore, by what we know from given and part $1$, if $p(x) > 0$ for all $x \in [a,b]$, then $\int_{a}^{b} p(x)x^kdx > 0$ when $x^k > 0$ and $\int_{a}^{b} p(x)x^kdx < 0$ when $x^k < 0$. Same when $p(x) < 0$. Hence, the sign of $p(x)$ must changes in the interval $[a,b] \implies$ there must have a root in the $[a,b]$.\\
            
            \item $Q(x)$ is in the span$\{x_i\}_{i=0}^{i=n-1}$, in part $b$ we already proved that $p(x)$ is orthogonal to this span. Therefore, $\int_{a}^{b} p(x)Q(x)dx = 0$\\
            
            \item Assume that $p(x)$ has exactly $1$ root $x_i$ with loss of generality, if its multiplicity $m \geq 2$, $p(x) = Q(x)(x-x_i)^2$ where $Q(x) = Q'(x)(x-x_i)^{m-2}$.\\
            
            $\int_{a}^{b} p(x)Q(x)dx = \int_{a}^{b} Q(x)^2 (x-x_i)^2dx \geq 0$. Contradicts to the $\int_{a}^{b} p(x)Q(x)dx = 0$\\
            
            Hence, for a root $x_i \in [a,b]$, it must have multiplicity one.\\
            
            \item For any $k = 1, ..., n-1$, let $q_k(x) = (x - x_1)...(x - x_k)$\\
            
            We know that from part $d$ that $\int_{a}^{b} p(x)Q(x)dx = 0$, let $Q(x) = q_k(x)$ and $p(x) = q_k(x)R(x)$ where $R(x)$ is the remaining term. Then, $\int_{a}^{b} q(x)^2R(x)dx = 0$\\
            
            As $q(x) \geq 0$, then $R(x)$ must also has a root in $[a,b]$. Therefore, $p(x)$ will have at least one another root.\\
        \end{enumerate}
        
    \item  
        \begin{enumerate}
            \item $f(x) = q(x)p(x) + r(x) \implies \int_{a}^{b} f(x)dx = \int_{a}^{b} q(x)p(x)dx + \int_{a}^{b}r(x) dx$ \\
            
            $\int_{a}^{b} q(x)p(x)dx = 0 \implies \int_{a}^{b} f(x)dx = \int_{a}^{b}r(x) dx$\\
            
            \item We know that for Lagrange basis function, $l_i(x_i) = 1$ if $i = j$ and $l_i(x_i) = 0$ otherwise.\\
            
            $r(x) = \sum_{i=1}^{i=n}r(x_i)l_i(x_i) \implies \int_{a}^{b}r(x) = \sum_{i=1}^{i=n}r(x_i)\int_{a}^{b}l_i(x_i)$ (If two polynomial agrees on $n$ points and with degrees both less than $n-1$, this holds)\\
            
            \item Simply because that $x_i$ are the roots for $p(x)$, therefore, $f(x_i) = r(x_i)$ from $f(x) = q(x)p(x) + r(x)$.\\
            As $f(x)$ has the same root as the $p(x)$, same as $r(x)$, we can say that the n-point interpolatory quadrature rule on $[a,b]$ whose nodes are the zeros of $p(x)$ must has degree $2n - 1$
        \end{enumerate}
\end{enumerate}

\end{solution}
References: \\ 

\end{document}
