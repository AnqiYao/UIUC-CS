% ---------
%  Compile with "pdflatex hw0".
% --------
%!TEX TS-program = pdflatex
%!TEX encoding = UTF-8 Unicode

\documentclass[11pt]{article}
\usepackage{jeffe,handout,graphicx}
\usepackage[utf8]{inputenc}	% Allow some non-ASCII Unicode in source
\usepackage{fourier-orns}

\usepackage{enumerate}
\usepackage{stmaryrd}

\def\Cdot{\mathbin{\text{\normalfont \textbullet}}}
\def\Sym#1{\textbf{\texttt{\color{BrickRed}#1}}}


% =========================================================
\begin{document}

\thispagestyle{empty}

\begin{center}
\Large\textbf{CS/ECE 374 \,\decosix\,  Fall 2018}%
\\
\LARGE\textbf{\decothreeleft~ Homework 0 ~\decothreeright}%
\\[0.5ex]
\large Due Wednesday, September 5, 2018 at 10am
\end{center}

\bigskip
\hrule
\bigskip
\begin{itemize}

\item
\textbf{Each student must submit individual solutions for this homework.}   For all future homeworks, groups of up to three students can submit joint solutions.

\item \textbf{Submit your solutions electronically on the course
    Gradescope site as PDF files.}  Submit a separate PDF file for
  each numbered problem.  If you plan to typeset your solutions,
  please use the \LaTeX\ solution template on the course web site.  If
  you must submit scanned handwritten solutions, please use a black
  pen on blank white paper and a high-quality scanner app (or an
  actual scanner, not just a phone camera).

\item You are \EMPH{not} required to sign up on Gradescope (or Piazza)
  with your real name and your illinois.edu email address; you may use
  any email address and alias of your choice.  However, to give you
  credit for the homework, we need to know who Gradescope thinks you
  are.  \textbf{Please fill out the web form linked from the course
    web page.}

\end{itemize}

\bigskip
\hrule
\bigskip

\begin{center}
\Large \textbf{\color{Red}\lefthand~
	Some important course policies
	~\righthand}\\
\end{center}

\begin{itemize}
\item \textbf{You may use any source at your disposal}—paper,
  electronic, or human—but you \EMPH{must} cite \EMPH{every} source
  that you use, and you \EMPH{must} write everything yourself in your
  own words.  See the academic integrity policies on the course web
  site for more details.

\item Unlike some previous semesters we will \textbf{not} have the ``I
  Don't Know (IDK)'' policy this semester for home works or exams. 

\item \textbf{Avoid the Three Deadly Sins!}  Any homework or exam
  solution that breaks any of the following rules will be given an
  \textcolor{Red}{\EMPH{automatic zero}}, unless the solution is
  otherwise perfect.  Yes, we really mean it.  We’re not trying to be
  scary or petty (Honest!), but we do want to break a few common bad
  habits that seriously impede mastery of the course material.
\begin{itemize}\itemsep0pt
\item Always give complete solutions, not just examples.
\item Always declare all your variables, in English.  In particular, always describe the specific problem your algorithm is supposed to solve.
\item Never use weak induction.
\end{itemize}

\end{itemize}
\bigskip
\hrule

\begin{center}
\large \textbf{See the course web site for more information.}\\[1ex]\normalsize
If you have any questions about these policies,\\
please don’t hesitate to ask in class, in office hours, or on Piazza.
\end{center}

\hrule
\vfil
\vfil
\vfil
%----------------------------------------------------------------------
\headers{CS/ECE 374}{Homework 0 (due September 5)}{Fall 2018}

\noindent

\begin{enumerate}
\parindent 1.5em \itemsep 4ex plus 0.5fil

\item[0.] Read the course policies and the instructions for this home work
before you start.

%----------------------------------------------------------------------
\item Suppose $S$ is a set of $103$ integers. Prove that there is a
    subset $S' \subseteq S$ of at least $15$ numbers such that 
    the difference of any two numbers in $S'$ is a multiple of $7$.
%----------------------------------------------------------------------
\item Consider the following recurrence.
$$T(n) = T(\floor{n/2}) + 2T(\floor{n/4}) + n \quad n \ge 4, \mbox{~and~} T(n) = 1 \quad 1 \le n < 4. $$
\begin{itemize}
\item Prove by induction that $T(n) = O(n \log n)$.
More precisely show that $T(n) \le a n \log n + b$ for $n \ge 1$ where
$a,b \ge 0$ are some fixed but suitably chosen constants (you get to choose and fix them).

\item {\bf For your reading, do not submit for grading:} Prove by induction that $T(n) = \Omega(n \log n)$. That is, show
  that $T(n) \ge a n \log n - b$  for $n \ge 1$ where $a,b \ge 0$ are fixed but
  suitably chosen constants .

\item {\bf For your reading, do not submit for grading:} Consider the recurrence:
$$T(n) = T(\floor{n/2}) + 2T(\floor{n/4}) + n^2 \quad n \ge 4, \mbox{~and~} T(n) = 1 \quad 1 \le n < 4.$$
Prove by induction that $T(n) = O(n^2)$.
\item {\bf For your reading, do not submit for grading:} Consider the recurrence:
$$T(n) = T(\ceil{3n/4}) + n \quad n \ge 4, \mbox{~and~} T(n) = 10 \quad 1 \le n < 4.$$
Prove by induction that $T(n) = O(n)$.
\end{itemize}

%----------------------------------------------------------------------
\item Consider the set of strings $L \subseteq \set{\Sym0,\Sym1}^*$
  defined recursively as follows:
\begin{itemize}
\item The string $\Sym1$ is in $L$.
\item For any string $x$ in $L$, the string $\Sym0 x$ is also in $L$.
\item For any string $x$ in $L$, the string $x \Sym0$ is also in $L$.
\item For any strings $x$ and $y$ in $L$, the string $x \Sym1 y$ is also in $L$.
\item These are the only strings in $L$.
\end{itemize}

\begin{enumerate}
\item Prove by induction that every string $w \in L$ contains an odd
  number of \Sym1s.
\item
Is every string $w$ that contains an odd number of \Sym1s in $L$?
In either case prove your answer.
\end{enumerate}
Let $\#(a, w)$ denote the number of times symbol $a$ appears in string $w$; for example,
\[
	\#(\Sym0, \Sym{101110101101011}) = 5
	\quad\text{and}\quad
	\#(\Sym1, \Sym{101110101101011}) = 10.
\]
You may assume without proof that $\#(a, uv) = \#(a, u) + \#(a, v)$ for any symbol $a$ and any strings $u$ and $v$, or any other result proved in class, in lab, or in the lecture notes.  Otherwise, your proofs must be formal and self-contained.

\item {\bf For your reading, do not submit for grading:} Let $\Sigma$ be a \emph{finite} alphabet
and let $\mathcal{L}$ be the set of all {\em finite} languages over $\Sigma$.
Prove that $\mathcal{L}$ is countable. Note that Cantor's diagonalization
argument (review CS 173 material if you have forgotten about countability) shows
that if $|\Sigma| \ge 2$ the set of all languages over $\Sigma$ is 
\emph{not} countable.

\item {\bf For your reading, do not submit for grading:} 
The famous Basque computational arborist Gorka Oihanean has a favorite 26-node binary tree, in which each node is labeled with a letter of the alphabet.  Intorder and postorder traversals of his tree visits the nodes in the following orders:
\begin{align*}
	\text{Inorder:~}	& \Sym{F E V I B H N X G W A Z O D J S R M U T C K Q P L Y} \\
	\text{Postorder:~}	& \Sym{F V B I E N A Z W G X J S D M U R O H K C Q Y L P T}
\end{align*}

\begin{enumerate}
\item
List the nodes in Professor Oihanean’s tree according to a preorder traversal.

\item
Draw Professor Oihanean’s tree.

\end{enumerate}

\end{enumerate}

\newpage

\begin{boxquote}{0.95}
  Each homework assignment will include at least one solved problem,
  similar to the problems assigned in that homework, together with the
  grading rubric we would apply \emph{if} this problem appeared on a
  homework or exam.  These model solutions illustrate our
  recommendations for structure, presentation, and level of detail in
  your homework solutions.  Of course, the actual \emph{content} of
  your solutions won’t match the model solutions, because your
  problems are different!
\end{boxquote}

\subsection*{Solved Problems}
\begin{enumerate}
\item[4.] Suppose $S$ is a set of $n+1$ 
    integers.  Prove that there exist distinct numbers $x,y \in S$
    such that $x-y$ is a multiple of $n$. {\em Hint:} 

    \begin{solution}
      We will use the pigeon hole principle. Let the $n+1$ numbers in $S$
      be $a_1,a_2,\ldots,a_{n+1}$ and consider $b_1,b_2,\ldots,b_{n+1}$ where
      $b_i = a_i \mod n$. Note that each $b_i$ belongs to the
      set $\{0,1,\ldots,n-1\}$. By the pigeon hole principle we must have
      two numbers $b_i$ and $b_j$, $i \neq j$ such that $b_i = b_j$.
      This implies that $a_i \mod n = a_j \mod n$ and hence $a_i - a_j$ is
      divisible  by $n$.

      \begin{rubric}
        2 points for recognizing that the pigeon hole principle can be
        used. 2 points for the idea of using $\mod n$. 6 points for a
        full correct proof. Any other correct proof would also fetch 10 points.
\end{rubric}

    \end{solution}

\item[5.]
Recall that the \EMPH{reversal $w^R$} of a string $w$ is defined recursively as follows:
\[	
	w^R := \begin{cases}
		\e & \text{if $w=\e$} \\[0.5ex]
		x^R \Cdot a & \text{if $w=a\cdot x$}
	\end{cases}
\]
A \EMPH{palindrome} is any string that is equal to its reversal, like \Sym{AMANAPLANACANALPANAMA}, \hbox{\Sym{RACECAR}}, \Sym{POOP}, \Sym{I}, and the empty string.
\begin{enumerate}
\item
Give a recursive definition of a palindrome over the alphabet $\Sigma$.

\item
Prove $w = w^R$ for every palindrome $w$ (according to your recursive definition).  

\item
Prove that every string $w$ such that $w = w^R$ is a palindrome (according to your recursive definition).

\end{enumerate}
In parts (b) and (c), you may assume without proof that $(x\cdot y)^R = y^R\Cdot x^R$ and $(x^R)^R = x$ for all strings $x$ and~$y$.



\begin{solution}~
\begin{enumerate}
\item
A string $w\in\Sigma^*$ is a palindrome if and only if either
\begin{itemize}
\item $w = \e$, or
\item $w = a$ for some symbol $a\in\Sigma$, or
\item $w = axa$ for some symbol $a\in\Sigma$ and some \emph{palindrome} $x\in\Sigma^*$.
\end{itemize}

\begin{rubric}
2 points = \textonehalf\ for each base case + 1 for the recursive case.  No credit for the rest of the problem unless this is correct.
\end{rubric}


\medskip
\item
Let $w$ be an arbitrary palindrome.

Assume that $x = x^R$ for every palindrome $x$ such that $\abs{x}<\abs{w}$.

There are three cases to consider (mirroring the three cases in the definition):
\begin{itemize}
\item
If $w = \e$, then $w^R = \e$ by definition, so $w = w^R$.

\item
If $w = a$ for some symbol $a\in\Sigma$, then $w^R = a$ by definition, so $w = w^R$.

\item
Suppose $w = axa$ for some symbol $a\in\Sigma$ and some palindrome $x\in P$.  Then 
\begin{align*}
	w^R
	&=	(a \cdot x \Cdot a)^R		\\
	&=	(x\Cdot a)^R \Cdot a		& \text{by definition of reversal} \\
	&=	a^R \Cdot x^R \Cdot a		& \text{You said we could assume this.}\\
	&=	a \Cdot x^R \Cdot a			& \text{by definition of reversal} \\
	&=	a \Cdot x \Cdot a			& \text{by the inductive hypothesis} \\
	&=	w							& \text{by assumption}
\end{align*}
\end{itemize}
In all three cases, we conclude that $w = w^R$.

\begin{rubric}
4 points: standard induction rubric (scaled)
\end{rubric}

\medskip
\item
Let $w$ be an arbitrary string such that $w = w^R$.

Assume that every string $x$ such that $\abs{x} < \abs{w}$ and $x = x^R$ is a palindrome.

There are three cases to consider (mirroring the definition of “palindrome”):
\begin{itemize}
\item 
If $w = \e$, then $w$ is a palindrome by definition.
\item 
If $w = a$ for some symbol $a\in\Sigma$, then $w$ is a palindrome by definition.
\item
Otherwise, we have $w = ax$ for some symbol $a$ and some \emph{non-empty} string $x$.
  
The definition of reversal implies that $w^R = (ax)^R = x^R a$.

Because $x$ is non-empty, its reversal $x^R$ is also non-empty.

Thus, $x^R = by$ for some symbol $b$ and some string~$y$.

It follows that $w^R = bya$, and therefore $w = (w^R)^R = (bya)^R = a y^R b$.


\medskip
\emph{[At this point, we need to prove that $a=b$ and that $y$ is a palindrome.]}
\medskip

Our assumption that $w = w^R$ implies that $bya = a y^R b$.

The recursive definition of string equality immediately implies $a=b$.

\medskip
Because $a=b$, we have $w = ay^Ra$ and $w^R = a y a$.

The recursive definition of string equality implies $y^Ra = ya$.

It immediately follows that $(y^R a)^R = (ya)^R$.

Known properties of reversal imply $(y^R a)^R = a (y^R)^R = ay$ and $(ya)^R = a y^R$.

It follows that $ay^R = ay$, and therefore $y = y^R$.

The inductive hypothesis now implies that $y$ is a palindrome.

\medskip
We conclude that $w$ is a palindrome by definition.
\end{itemize}
In all three cases, we conclude that $w$ is a palindrome.
\end{enumerate}

\begin{rubric}
4 points: standard induction rubric (scaled).
\begin{itemize}
\item No penalty for jumping from $aya = ay^Ra$ directly to $y = y^R$.
\end{itemize}
\end{rubric}

\end{solution}



\begin{rubric}[induction]
For problems worth 10 points:
\begin{itemize}\itemsep0pt
\item[+] 1 for explicitly considering an \emph{arbitrary} object

\item[+] 2 for a valid \textbf{strong} induction hypothesis
\begin{itemize}\itemsep0pt
\item \textbf{\color{Red}Deadly Sin!} Automatic zero for stating a weak induction hypothesis, unless the rest of the proof is \emph{perfect}.
\end{itemize}

\item[+] 2 for explicit exhaustive case analysis
\begin{itemize}\itemsep0pt
\item No credit here if the case analysis omits an infinite number of objects.  (For example: all odd-length palindromes.)
\item $-1$ if the case analysis omits an finite number of objects.  (For example: the empty string.)
\item $-1$ for making the reader infer the case conditions.  Spell them out!
\item No penalty if cases overlap (for example:
\end{itemize}

\item[+] 1 for cases that do not invoke the inductive hypothesis (“base cases”)
\begin{itemize}\itemsep0pt
\item No credit here if one or more “base cases” are missing.
\end{itemize}

\item[+] 2 for correctly applying the \emph{stated} inductive hypothesis
\begin{itemize}\itemsep0pt
\item No credit here for applying a \emph{different} inductive hypothesis, even if that different inductive hypothesis would be valid.
\end{itemize}

\item[+] 2 for other details in cases that invoke the inductive hypothesis (“inductive cases”)
\begin{itemize}\itemsep0pt
\item No credit here if one or more “inductive cases” are missing.
\end{itemize}
\end{itemize}
\end{rubric}

\end{enumerate}

\end{document}
