% ---------
%  Compile with "pdflatex hw0".
% --------
%!TEX TS-program = pdflatex
%!TEX encoding = UTF-8 Unicode

\documentclass[11pt]{article}
\usepackage{jeffe,handout,graphicx}
\usepackage[utf8]{inputenc}		% Allow some non-ASCII Unicode in source

%  Redefine suits
\usepackage{pifont}
\def\Spade{\text{\ding{171}}}
\def\Heart{\text{\textcolor{Red}{\ding{170}}}}
\def\Diamond{\text{\textcolor{Red}{\ding{169}}}}
\def\Club{\text{\ding{168}}}

\def\Cdot{\mathbin{\text{\normalfont \textbullet}}}
\def\Sym#1{\textbf{\texttt{\color{BrickRed}#1}}}



% =====================================================
%   Define common stuff for solution headers
% =====================================================
\Class{CS/ECE 374}
\Semester{FALL $2018$}
\Authors{3}
\AuthorOne{Zhe Zhang}{zzhan157@illinois.edu}
\AuthorTwo{Ray Ying}{xinruiy2@illinois.edu}
\AuthorThree{Anqi Yao}{anqiyao2@illinois.edu}
%\Section{}
% =====================================================
\documentclass[12pt]{article}
\usepackage{tikz}
\begin{document}

% ---------------------------------------------------------

\HomeworkHeader{3}{3}

\begin{quote}

\item Given languages $L_1$ and $L_2$ we define $\text{\em insert}(L_1,L_2)$
to be the language $\{uvw \mid v \in L_1, uw \in L_2\}$ to be the
set of strings obtained by ``inserting'' a string of $L_1$ into a
string of $L_2$. For example if $L_1 = \{isfun\}$ and $L_2 = \{0, CS\}$
then
$$\text{\em insert}(L_1,L_2) = \{isfun0,0isfun,isfunCS,CisfunS,CSisfun\}$$
\begin{itemize}
\item The goal is to show that if $L_1$ and $L_2$ are regular
  languages then $\text{\em insert}(L_1,L_2)$ is also regular.  In
  particular you should describe how to construct an NFA $N =
  (Q,\Sigma, \delta,s, A)$ from two DFAs
  $M_1=(Q_1,\Sigma,\delta_1,s_1,A_1)$ and
  $M_2=(Q_2,\Sigma,\delta_2,s_2,A_2)$ such that $L(N) = \text{\em
    insert}(L(M_1),L(M_2))$. You do not need to prove the correctness of
  your construction but you should explain the ideas behind the
  construction (see lab 3 solutions).
\end{itemize}
\end{quote}
\hrule


\begin{solution}
\item
\begin{enumerate}
\item   Let $M_1=(Q_1,\Sigma,\delta_1,s_1,A_1)$ and $M_2=(Q_2,\Sigma,\delta_2,s_2,A_2)$ such that $L(N) = \text{\em
    insert}(L(M_1),L(M_2))$, We construct an NFA $N =
  (Q,\Sigma, \delta,s, A)$ that accepts ${\em
    insert}(L(M_1),L(M_2))$ as follows: \newline
    ${\em Q} := Q_1 * Q_2$ \newline
    ${\em \Sigma} := \Sigma$ \newline
    ${\em \delta((q_2, before), a)} := \left\{\begin{array}{l}$\{\delta((q_2, a), before), (q_2,after)\}$   a $\in$ L(M_1)\\ \{\delta((q_2, a), before)\}
    otherwise\end{array}\right.$ \newline
    ${\em \delta((q_2, after), a)} := \{\delta((q_2, a), after)\}$ a $\in$ $L(M_2)$\newline 
    ${\em s} := \{s_1, s_2\}$ \newline
    ${\em A} := A_1 \cap A_2 $\newline
    For the states in the NFA, $Q$ is going to be $Q_1 * Q_2$ without any doubts. Since we can start at both old state $s_1$ and $s_2$, the start state for NFA is just the set contain both $s_1$ and $s_2$. Same reason apply for accepting state, we can end at both old accepting states, the accepting states for NFA is the union of $A_1$ and $A_2$. As for the language it accepts, doesn't matter if the input language that start with $L(M_1)$ or $L(M_2)$, once it receive the language in $L(M_1)$, because $L(N) = \text{\em
    insert}(L(M_1),L(M_2))$, only breaks the language that is in $L(M_2)$ in half, that means once we pass or enter the start state in $L(M_1)$, there is only one way to the accepting state (only one accepting state as there is only one route).
    \begin{itemize}
        \item The state ($q_2$, before) means (the simulation of) $M_2$ is in state $q_2$ and $N$ has not yet enter the start state for $M_1$.
        \item The state ($q_2$, after) means (the simulation of) $M_2$ is in state $q_2$ and $N$ has passed the accepting state for $M_1$.
    \end{itemize}
\end{enumerate}


\end{solution}



\end{document}
