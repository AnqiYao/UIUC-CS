% ---------
%  Compile with "pdflatex hw4".
% --------
%!TEX TS-program = pdflatex
%!TEX encoding = UTF-8 Unicode

\documentclass[11pt]{article}
\usepackage{jeffe,handout,graphicx}
\usepackage[utf8]{inputenc}		% Allow some non-ASCII Unicode in source

%  Redefine suits
\usepackage{pifont}
\def\Spade{\text{\ding{171}}}
\def\Heart{\text{\textcolor{Red}{\ding{170}}}}
\def\Diamond{\text{\textcolor{Red}{\ding{169}}}}
\def\Club{\text{\ding{168}}}

\def\Cdot{\mathbin{\text{\normalfont \textbullet}}}
\def\Sym#1{\textbf{\texttt{\color{BrickRed}#1}}}



% =====================================================
%   Define common stuff for solution headers
% =====================================================
\Class{CS/ECE $374$}
\Semester{FALL $2018$}
\Authors{3}
\AuthorOne{Zhe Zhang}{zzhan157@illinois.edu}
\AuthorTwo{Ray Ying}{xinruiy2@illinois.edu}
\AuthorThree{Anqi Yao}{anqiyao2@illinois.edu}
%\Section{}
% =====================================================
\documentclass[12pt]{article}
\usepackage{tikz}
\begin{document}

% ---------------------------------------------------------

\HomeworkHeader{$6$}{$1$}

\begin{enumerate}[$1$.]
\item The McKing chain wants to open several restaurants along Red
  street in Shampoo-Banana. The possible locations are at $L_1,L_2,
  \ldots, L_n$ where $L_i$ is at distance $m_i$ meters from the start
  of Red street. Assume that the street is a straight line and the
  locations are in increasing order of distance from the starting
  point (thus $0 \leq m_1 < m_2 < \ldots < m_n$). McKing has collected
  some data indicating that opening a restaurant at location $L_i$
  will yield a profit of $p_i$ independent of where the other
  restaurants are located. However, the city of Shampoo-Banana has a
  zoning law which requires that any two McKing locations should be
  $D$ or more meters apart. {\em In addition McKing does not want to
    open more than $k$ restaurants due to budget constraints.}
  Describe an efficient algorithm that McKing can use to figure out the maximum
  profit it can obtain by opening restaurants while satisfying the
  city's zoning law and the constraint of opening at most $k$
  restaurants.  Your algorithm should use only $O(n)$ space and you
  should not assume that $k$ is a constant.

\end{enumerate}

\hrule


\begin{solution} (Reference Lab 8) \\
To simplify boundary cases, let's add a location $L_{n+1}$ where $L_{n+1}$ is at distance $m_{n+1} = \infty$. \\\\
For any index i, let MaxProfit(i,k) denotes the maximum profit McKing can obtain by opening restaurants on locations $L_i,L_{i+1}, \ldots, L_n$ while satisfying the city's zoning law and the constraint of opening at most k restaurants. \\\\
For any index i, let next(i) denotes the smallest index j such that $m_j$ > $m_i$ + D. The possible locations are at $L_1,L_2, \ldots, L_n$. For each possible location $L_i$, there are two cases: (1) we open one restaurant and open the rest $k-1$ restaurants on other possible locations $L_{i+1}, \ldots, L_n$. (2) we do not open one restaurant and open $k$ restaurants on other possible locations $L_{i+1}, \ldots, L_n$. We will follow the case which gives us the maximum profit. The function MaxProfit(i,k) statisfies the following recurrence: \\
\begin{equation}
  \text{MaxProfit}(i,k)=\begin{cases}
    0, & \text{if $i>n$ or $k=0$}.\\
    max\{p_i+\text{MaxProfit}(\text{next}(i),k-1),\text{MaxProfit}(i+1,k)\}, & \text{otherwise}.
  \end{cases}
\end{equation}
We can memorize this function into a two-dimensional array MaxProfit[$1$...n+1,$0$...k]. Because the array $m_1, m_2, \ldots, m_n$ is sorted, we can compute next(i) for any index i in O(logn) time using binary search. There are O(nk) distinct subproblems so that the total running time is O(nklogn). Space for the aray is O(kn). However, for each subproblem in column k, only column k and column k-1 matter so that instead of memorizing k columns, it is sufficient to memorize only two columns. Then the space for the array is O(n).
  



\end{solution}

\end{document}