% ---------
%  Compile with "pdflatex hw4".
% --------
%!TEX TS-program = pdflatex
%!TEX encoding = UTF-8 Unicode

\documentclass[11pt]{article}
\usepackage{jeffe,handout,graphicx}
\usepackage[utf8]{inputenc}		% Allow some non-ASCII Unicode in source

%  Redefine suits
\usepackage{pifont}
\def\Spade{\text{\ding{171}}}
\def\Heart{\text{\textcolor{Red}{\ding{170}}}}
\def\Diamond{\text{\textcolor{Red}{\ding{169}}}}
\def\Club{\text{\ding{168}}}

\def\Cdot{\mathbin{\text{\normalfont \textbullet}}}
\def\Sym#1{\textbf{\texttt{\color{BrickRed}#1}}}



% =====================================================
%   Define common stuff for solution headers
% =====================================================
\Class{CS/ECE 374}
\Semester{FALL $2018$}
\Authors{3}
\AuthorOne{Zhe Zhang}{zzhan157@illinois.edu}
\AuthorTwo{Ray Ying}{xinruiy2@illinois.edu}
\AuthorThree{Anqi Yao}{anqiyao2@illinois.edu}
%\Section{}
% =====================================================
\documentclass[12pt]{article}
\usepackage{tikz}
\begin{document}

% ---------------------------------------------------------

\HomeworkHeader{$4$}{$2$}

\begin{enumerate}[2.]
\item Suppose you are given $k$ sorted arrays $A_1,A_2,\ldots,A_k$
  each of which has $n$ numbers. Assume that all numbers in the arrays
  are distinct. You would like to merge them into single sorted array
  $A$ of $kn$ elements. Recall that you can merge two sorted arrays of
  sizes $n_1$ and $n_2$ into a sorted array in $O(n_1+n_2)$ time.
  \begin{itemize}
  \item Use a divide and conquer strategy to merge the sorted arrays
    in $O(nk \log k)$ time. To prove the correctness of the algorithm
    you can assume a routine to merge two sorted arrays.
  \item In MergeSort we split the array of size $N$ into two arrays
    each of size $N/2$, recursively sort them and merge the two sorted
    arrays. Suppose we instead split the array of size $N$ into $k$
    arrays of size $N/k$ each and use the merging algorithm in the
    preceding step to combine them into a sorted array.  Describe the
    algorithm formally and analyze its running time via a recurrence.
    You do not need to prove the correctness of the recursive algorithm.
  \end{itemize}
\end{enumerate}

\hrule


\begin{solution}
1.We could merge the first array $A_1$ with the second array $A_2$, the third array $A_3$ with the fourth array $A_4$ and so on. Then k sorted arrays will become $k/2$ sorted arrays. We will do this procedure recursively until there is only $1$ sorted array containing kn elements.
\newline
\newline
The algorithm can be represented by the following pseudocode:

\begin{quote}
        \newline
        \fbox{\begin{minipage}{30em}
        mergeArrays($A_1,A_2,...,A_k$):
        \quote
        if(numberof($A_1,A_2,...,A_k$) == 1) 
        \newline
        \hspace*{2ex} return $A_1$
        \newline
        firstpart = mergeArrays($A_1,A_2,...,A_{k/2}$)
        \newline
        secondpart = mergeArrays($A_{k/2+1},A_{k/2+2},...,A_k$)
        \newline
        result = merge(firstpart, secondpart)
        \newline
        return result
        \end{minipage}}
\end{quote}
Each recursive step we do $O(kn)$ work to merge the k sorted arrays into $k/2$ sorted arrays. Since we will continue doing this procedure until there is only $1$ sorted array, we have to do $O(kn)$ work for $O(logk)$ times. Thus this algorithm takes $O(nklogk)$.
\newline
\newline
We will prove the correctness of the algorithm by doing induction on the number of sorted arrays, k.
\begin{itemize}
\item Base case: when k = 1, the algorithm returns a sorted array which is correct.
\item Inductive hypothesis: Suppose our algorithm holds for all the x such that $|x|<|k|$. The algorithm could merge x sorted arrays into a single sorted array containing xn elements.
\item Since both firstpart and secondpart in our algorithm contain $k/2$ elements where $|k/2|<|k|$, by the inductive hypothesis, both firstpart and secondpart are sorted arrays containing $kn/2$ elements. Since it is a routine to merge two sorted arrays, merge() will merge firstpart and secondpart into a single sorted array containing kn elements.
\end{itemize}
Thus our algorithm is correct.
\newline
\newline
2.
The algorithm can be represented by the following pseudocode:
\begin{quote}
        \newline
        \fbox{\begin{minipage}{30em}
        arrays[1],arrays[2],...,arrays[k]
        mergeSort($A[1,2,...,N]$):
        \quote
        if(length($A[1,2,...,N]$) == 1) 
        \newline
        \hspace*{2ex} return A
        \newline
        for(i in 1 : (k-1))
        \newline
        \hspace*{2ex} shift = (i-1) * (N/k)
        \newline
        \hspace*{2ex} arrays[i] = mergeSort(A[shift+1,shift+2,...,shift+N/k]
        \newline
        arrays[k] = mergeSort(A[(k-1)*(N/k)+1,...,N]
        \newline
        result = mergeArrays(arrays[1],arrays[2],...,arrays[k])
        \newline
        return result
        \end{minipage}}
\end{quote}
\newline
\newline
We know from part(1) that merging k sorted arrays which are array[1],array[2],...,array[k], needs O(Nlogk).Then We analyze the running time via a recurrence.
\newline
\newline
T(k)=\left\{
\begin{array}{c l}	
     O(1) & N = 1\\
     kT(N/k) + O(Nlogk) & otherwise
\end{array}\right
\newline
\newline
\text{The overall running time is O(NlogN).}

\end{solution}

\end{document}