% ---------
%  Compile with "pdflatex hw4".
% --------
%!TEX TS-program = pdflatex
%!TEX encoding = UTF-8 Unicode

\documentclass[11pt]{article}
\usepackage{jeffe,handout,graphicx}
\usepackage[utf8]{inputenc}		% Allow some non-ASCII Unicode in source

%  Redefine suits
\usepackage{pifont}
\def\Spade{\text{\ding{171}}}
\def\Heart{\text{\textcolor{Red}{\ding{170}}}}
\def\Diamond{\text{\textcolor{Red}{\ding{169}}}}
\def\Club{\text{\ding{168}}}

\def\Cdot{\mathbin{\text{\normalfont \textbullet}}}
\def\Sym#1{\textbf{\texttt{\color{BrickRed}#1}}}



% =====================================================
%   Define common stuff for solution headers
% =====================================================
\Class{CS/ECE $374$}
\Semester{FALL $2018$}
\Authors{3}
\AuthorOne{Zhe Zhang}{zzhan157@illinois.edu}
\AuthorTwo{Ray Ying}{xinruiy2@illinois.edu}
\AuthorThree{Anqi Yao}{anqiyao2@illinois.edu}
%\Section{}
% =====================================================
\documentclass[12pt]{article}
\usepackage{tikz}
\begin{document}

% ---------------------------------------------------------

\HomeworkHeader{$7$}{$3$}
\item Graphs are a powerful tool to model many phenomena.  The edges
  of a graph model pairwise relationships. It is natural to consider
  higher order relationships.  Indeed {\em hypergraphs} provide one
  such modeling tool.  A hypergraph $G=(V,\mathcal{E})$ consists of a
  finite set of nodes/vertices $V$ and a finite set of hyper-edges
  $\mathcal{E}$. A hyperedge $e \in \mathcal{E}$ is simply a subset of
  nodes and the cardinality of the subset can be larger than two. An
  undirected graph is a hypergraph where each hyper-edge is of size
  exactly two. Here is an example. $V=\{1,2,3,4,5\}$ and $\mathcal{E}
  = \{\{1,2\},\{2,3,4\},\{1,3,4,5\},\{2,5\}\}$. The representation
  size of a hypergraph is $|V| + \sum_{e \in \mathcal{E}} |e|$.  An
  alternating sequence of nodes and edges
  $x_1,e_1,x_2,e_2,\ldots,e_{k-1},x_k$ where $x_i \in V$ for $1 \le i
  \le k$ and $e_j \in \mathcal{E}$ for $1 \le j \le k-1$ is called a
  path from $u$ to $v$ if (i) $x_1 = u$ and $x_k = v$ and (ii) for $1
  \le j < k$, $x_j \in e_j$ and $x_{j+1} \in e_j$.

  \begin{itemize}
  \item Given a hypergraph $G=(V,\mathcal{E})$ and two nodes $u,v \in V$
    we say that $u$ is connected to $v$ if there is path from $u$ to $v$.
    We say that a hypergraph is connected
    if each pair of nodes $u,v$ in $G$ are connected. Describe an algorithm that
    given a hypergraph $G$ checks whether $G$ is connected in linear time.
    In essense describe a reduction of this problem to the standard graph
    connectivity problem. You need to prove the correctness of your algorithm.
  \item Suppose we want to quickly spread a message from one person to
    another person during an emergency on a social network called
    AppsWhat which is organized as a collection of groups. Messages
    sent by a group member are broadcast to the entire group. AppsWhat
    knows the members and the list of groups on its service. The goal
    is to find the fewest messages that need to be sent such that a
    person $u$ can reach a person $v$. Model this problem using
    hypergraphs and describe a linear-time algorithm for it.
    No proof necessary for this part.
  \end{itemize}
\hrule


\begin{solution}
$$  $$
\begin{enumerate}
    \item  
        We can use a single additional vertex represent a hyperedge in the graph $G$. Hence, We can construct a indirect graph $G'$ to convert the hypergraph $G = (V,\mathcal{E})$ into a standard graph where graph $G' = (V',E'):$
        \newline
        $ V' = V \cap \mathcal{E}$
        \newline
        $ E' = \{(v,e) | v\in V, e \in \mathcal{E},v \in e \}$
        \newline
        The running time construct the graph $G'$ will be $O(|V| + |\mathcal{E}|$ + $\sum_{e \in \mathcal{E}} |e|)$, which is in linear-time. 
        \newline 
        After constructing the graph $G'$, we can start at an arbitrary vertex in the graph $G'$ and using DFS algorithm to find the total number of vertex it connects to. We can compare it to $|V|$ afterwards. If it's the same, return true. Otherwise, false. 
        \newline
        The proof for this algorithm:
        \newline
        From the definition of hypergraph, An alternating sequence of nodes and edges 
        $$x_1,e_1,x_2,e_2,\ldots,e_{k-1},x_k$$
        where $x_i \in V$ for $1 \le i \le k$ and $e_j \in \mathcal{E}$ for $1 \le j \le k-1$ is called a path from $u$ to $v$ if (i) $x_1 = u$ and $x_k = v$ and (ii) for $1 \le j < k$, $x_j \in e_j$ and $x_{j+1} \in e_j$. By the definition of $G'$, $e_j$ in the old path is $(v_j,e_j)$ and there is vertex for each edge as well, each $e_j$ will be a vertex representing an edge. So the path for $u$ to $v$ in $G'$ will be 
        $$x_1,(x_1,e_1),e_1,(x_2,e_1),x_2,(x_2,e_2),e_2,...,x_k$$
        Also, when giving the path from $u$ to $v$ in $G'$ in the form of 
        $$x_1,(x_1,e_1),e_1,(x_2,e_1),x_2,(x_2,e_2),e_2,...,x_k$$
        To transfer to the path in $G$, since $(x_j,e_j)$ are edges in $G'$ and $e_1$ are not vertex but edge, so the path from $u$ to $v$ in graph $G$ can only be 
        $$x_1,e_1,x_2,e_2,\ldots,e_{k-1},x_k$$.
        Hence, we proved the correctness of the algorithm. The running time for the algorithm has the constructing time and DFS time. Both will be in linear time as we just discussed that constructing time is linear, so the total time complexity will be $O(|V| + \sum_{e \in \mathcal{E}} |e|)$.
    \item
        So basically, we want to think a hypergraph $G$ with each person as a vertex and each group as a hyperedge. So the question for the fewest message for person $u$ to reach person $v$ will be finding the shortest path between $u$ and $v$ in the hypergraph $G$. Then we can construct a graph $G'$ the same way in part $1$ for the hypergraph $G$. We will use a BFS algorithm on it to find the shortest path between $u$ and $v$. After we find the value $k$, we can divided $k$ in half. Since as we constructed in part $1$, one edge in hypergraph will be transformed into $2$ edges. So the fewest message needed to be sent is $k/2$. The running time for this algorithm is $O(|V| + \sum_{e \in \mathcal{E}} |e|)$ since both constructing graph and BFS are in linear time.
\end{enumerate}
\end{solution}
\end{document}