% ---------
%  Compile with "pdflatex hw4".
% --------
%!TEX TS-program = pdflatex
%!TEX encoding = UTF-8 Unicode

\documentclass[11pt]{article}
\usepackage{jeffe,handout,graphicx}
\usepackage[utf8]{inputenc}		% Allow some non-ASCII Unicode in source
\graphicspath{ {./Images/} }
%  Redefine suits
\usepackage{pifont}
\def\Spade{\text{\ding{171}}}
\def\Heart{\text{\textcolor{Red}{\ding{170}}}}
\def\Diamond{\text{\textcolor{Red}{\ding{169}}}}
\def\Club{\text{\ding{168}}}

\def\Cdot{\mathbin{\text{\normalfont \textbullet}}}
\def\Sym#1{\textbf{\texttt{\color{BrickRed}#1}}}



% =====================================================
%   Define common stuff for solution headers
% =====================================================
\Class{CS/ECE 374}
\Semester{FALL $2018$}
\Authors{3}
\AuthorOne{Zhe Zhang}{zzhan157@illinois.edu}
\AuthorTwo{Ray Ying}{xinruiy2@illinois.edu}
\AuthorThree{Anqi Yao}{anqiyao2@illinois.edu}
%\Section{}
% =====================================================
\documentclass[12pt]{article}
\usepackage{tikz}
\begin{document}

% ---------------------------------------------------------

\HomeworkHeader{$4$}{$1$}
\begin{enumerate}[1.]
\item A {\bf two-dimensional} Turing machine (2D TM for short) uses an infinite
  two-dimensional grid of cells as the tape. For simplicity assume
  that the tape cells corresponds to integers $(i,j)$ with $i,j \ge 0$;
  in other words the tape corresponds to the positive quadrant of the
  two dimensional plane. The machine crashes if it tries to move below
  the $x=0$ line or to the left of the $y=0$ line.  The transition
  function of such a machine has the form
  $\delta: Q \times \Gamma \rightarrow Q \times \Gamma \times
  \{L,R,U,D,S\}$ where $L$, $R$, $U$, $D$ stand for ``left'',
  ``right'', ``up'' and ``down'' respectively, and $S$ stands for
  ``stay put''. You can assume that the input to the 2D TM is written
  on the first row and that its head is initially at location $(0,0)$.
  Argue that a 2D TM can be simulated by an
  ordinary TM (1D TM); it may help you to use a multi-tape TM for
  simulation. In particular address the following points.
  \begin{itemize}
  \item How does your TM store the grid cells of a 2D TM on a one dimensional
    tape?
  \item How does your TM keep track of the head position of the
    2D TM?
  \item How does your 1D TM simulate one step of the
    2D TM?
  \end{itemize}
  If a 2D TM takes $t$ steps on some input how many steps
  (asymptotically) does your simulating 1D TM take on the same input?
  Give an asymptotic estimate. Note that it is quite difficult to give
  a formal proof of the simulation argument, hence we are looking for
  high-level arguments similar to those we gave in lecture for various
  simulations.

\end{enumerate}
\hrule


\begin{solution}
\item   
    The question asks us to show a $2D$ TM can be simulated by an ordinary TM($1D$ TM) $M'$ and the transition function for the $2D$ TM has the form: 
    
    $$\delta: Q \times \Gamma \rightarrow Q \times \Gamma \times  \{L,R,U,D,S\}$$ 
    
    where $L$, $R$,  ``up'', ``down'' and ``stay put''. It's also stat$U$, $D$, $S$ stands for ``left'', ``right'',ed that the initial location for the head is $(0,0)$. We can build the a one-way infinite (grid) tapes and have head starts at location $0$. 
    $$     $$ 
    $i$ and $j$ are the corresponding indexes on the each dimension in $2D$ TM. We can using the standard diagonal numbering function to determine the position of that grid cells in the one-way grid tape:
    $$ f(i,j) = \frac{(i+j)(i+j+1)}{2} + j $$
    Let the calculation for $f(i,j)$ done in another "scratch" tape. In this structure, we successfully store the grid cells of the $2D$ TM on a one dimensional tape. Compute the result $x = f(i,j)$ and then move the head of the grid tape to the $x_$th cell.(Mark a special mark on the start(leftmost) cell since this is a one-way infinite TM). Now, $1$ step in $2D$ TM will be similar to $1D$ TM that read a symbol on the current state, determine the next state and overwrite the current cell the same symbol how $2D$ TM overwrites its own current cell. Move the head base on the function $f(i,j)$ since $i$ or $j$ changes in the move of $2D$ TM. 
    $$ $$
    If a $2D$ TM takes $t$ steps, from the function $f(i,j) = \frac{(i+j)(i+j+1)}{2} + j$, the steps for $1D$ TM will be affected by $(i+j)$ and $j$. 
    Each step will result in $4$ cases$\{i+1, j+1, i-1, j-1\}$: 
    $$ (i+j+1) $$
    $$ (i+j+1) + 1 = (i+j+2) $$
    $$ |-(i+j)| = (i+j)$$
    $$ |-(i+j+1)| = (i+j+1)$$
    Take the largest one and times $t$ would be: $(i+j+2)*t$. Since the initial position is $(0,0)$, the asymptotic estimation would be: $2t$.
\end{solution}
\end{document}
