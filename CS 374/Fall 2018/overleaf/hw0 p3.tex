% ---------
%  Compile with "pdflatex hw0".
% --------
%!TEX TS-program = pdflatex
%!TEX encoding = UTF-8 Unicode

\documentclass[11pt]{article}
\usepackage{jeffe,handout,graphicx}
\usepackage[utf8]{inputenc}		% Allow some non-ASCII Unicode in source

%  Redefine suits
\usepackage{pifont}
\def\Spade{\text{\ding{171}}}
\def\Heart{\text{\textcolor{Red}{\ding{170}}}}
\def\Diamond{\text{\textcolor{Red}{\ding{169}}}}
\def\Club{\text{\ding{168}}}

\def\Cdot{\mathbin{\text{\normalfont \textbullet}}}
\def\Sym#1{\textbf{\texttt{\color{BrickRed}#1}}}



% =====================================================
%   Define common stuff for solution headers
% =====================================================
\Class{CS/ECE 374}
\Semester{Fall 2018}
\Authors{1}
%\Section{}

% =====================================================
\begin{document}

% ---------------------------------------------------------


% ---------------------------------------------------------
% Change authors again
\AuthorOne{Xinrui Ying}{xinruiy2@illinois.edu}


\HomeworkHeader{$0$}{$3$}

\begin{quote}
%----------------------------------------------------------------------
\item Consider the set of strings $L \subseteq \set{\Sym0,\Sym1}^*$
  defined recursively as follows:
\begin{itemize}
\item The string $\Sym1$ is in $L$.
\item For any string $x$ in $L$, the string $\Sym0 x$ is also in $L$.
\item For any string $x$ in $L$, the string $x \Sym0$ is also in $L$.
\item For any strings $x$ and $y$ in $L$, the string $x \Sym1 y$ is also in $L$.
\item These are the only strings in $L$.
\end{itemize}

\begin{enumerate}[(a)]
\item Prove by induction that every string $w \in L$ contains an odd
  number of \Sym1s.
\item
Is every string $w$ that contains an odd number of \Sym1s in $L$?
In either case prove your answer.
\end{enumerate}
Let $\#(a, w)$ denote the number of times symbol $a$ appears in string $w$; for example,
\[
	\#(\Sym0, \Sym{101110101101011}) = 5
	\quad\text{and}\quad
	\#(\Sym1, \Sym{101110101101011}) = 10.
\]
You may assume without proof that $\#(a, uv) = \#(a, u) + \#(a, v)$ for any symbol $a$ and any strings $u$ and $v$, or any other result proved in class, in lab, or in the lecture notes.  Otherwise, your proofs must be formal and self-contained.
%----------------------------------------------------------------------

\end{quote}
\hrule


\begin{solution}
\item 
\begin{enumerate}[(a)]
\item
\begin{itemize}
\item
    Base case: When $w$ length = $1$. The only string for $w$ is $1$, which has odd number of 1s. So the base case holds.
\item
    Induction step: Suppose $w$ having odd number of $1s$ is true for length $1,2,3...k$, we want to length = $k+1$ also holds. There are only the following three cases:
    \begin{itemize}
    \item  
        We already have $w$ with length $k$ having odd number of $1s$, since $w \in L$, we also have $w0 \in L$. The number of $1s$ in $w0$ is same as number of 1s in $w$, the length of $w0$ = length of $w$ + 1 = $k$ + 1. So $w$ holds for length $k+1$ in this case.
    \item  
        We already have $w$ with length $k$ having odd number of $1s$, since $w \in L$, we also have $0w \in L$. The number of $1s$ in $0w$ is same as number of 1s in $w$, the length of $0w$ = length of $w$ + 1 = $k$ + 1. So $w$ holds for length $k+1$ in this case.
    \item  
        Let $m \in  L$ be any string with length $k1$ where $k1 < k$, and $n \in L$ be any string with length $k2$ where $k2 = k - k1$. Since $m$ and $n$ have length less then $k$, we can conclude that they both have odd number of $1s$. Since both $m$ and $n$ are in L, then $m1n$ is also in L. The number of $1s$ in $m1n$ = number of $1s$ in $m$ + number of $1s$ in $n$ + 1. Since odd + odd + 1 = odd, we have number of $1s$ in $m1n$ is odd. The length of $m1n$ = $k2 + k1 + 1$ = $k + 1$. So $w$ holds for length $k+1$ in this case.  
     
\end{itemize}
    The prove of every string $w \in L$ contains an odd number of $1s$ is complete.
\end{itemize}
\item    
    It's true that every string $w$ contains an odd number of $1s$ is in $L$.
\begin{itemize}
\item
    Base case: $w$ contains only one $1s$. Let $w$ starts with just $1$, then $w$ is in $L$, you can concatenate $0$ at the front or the end to still be in $L$. After you concatenate for the first time, you still have $w$ in L and you can recursively doing this and you can have any number of $0s$ in the front or back of the $1$ and you still have $w \in L$. So the base case holds.
\item
    Induction step: Suppose it true for $w$ with $1,3,5...k-2,k$ number of $1s$, we want to prove that number of $1s$ = $k + 2$ also holds. Let $m$ be any string with $k$ number of $1s$, since the statement is assume true for $k$ number of $1s$, we have $m \in L$. Let $n$ be any string with $1$ number of $1s$, since the statement is proved true for $1$ number of $1s$, we have $n \in L$. Now since both $m$ and $n$ are in $L$, then $m1n$ and $n1m$ are both in $L$. The number of $1s$ for $m1n$ or $n1m$ = $1$ + $1$ + $k$ = $k+2$. Since $m$ and $n$ can be any string with just required number of $1s$, $m1n$ and $n1m$ contain all the possible outcomes with $k+2$ of $1s$. $m1n$ and $n1m$ are both in $L$, hence the statement is true for $k+2$ number of $1s$.
\end{itemize}
    The prove of every string $w$ contains an odd number of $1s$ is in L is complete.
\medskip
\end{enumerate}

\end{solution}
\end{document}
